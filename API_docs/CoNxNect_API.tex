\documentclass[a4paper,11pt]{article}
%\binoppenalty=9999%10000%9999
%\relpenalty=9999%10000%9999
%\interfootnotelinepenalty=9999

%\usepackage{showframe}
\usepackage{amsthm}
\usepackage{amsmath}
\usepackage{amssymb}
\usepackage{bbm}
\usepackage{mathtools}
\usepackage{hyperref}
%\usepackage{showlabels}
\usepackage{xcolor}
\usepackage{fancyhdr}
\usepackage[nottoc, notlot, notlof]{tocbibind}
\usepackage[longnamesfirst]{natbib}
\usepackage{pdfpages}
\usepackage{makecell}
%\usepackage{graphicx}

%\setcitestyle{authoryear, open={(},close={)}}


%My Macros
%custom commands
\newcommand{\sinc}{\mathrm{sinc}}
\newcommand{\supp}{\mathrm{supp}}
\newcommand{\st}{\mathrm{st}}
\newcommand{\conv}{\mathrm{conv}}
\newcommand{\dist}{\mathrm{dist}}
\newcommand{\INT}{\mathrm{int}}
\newcommand{\x}{\times}
\newcommand{\p}{\partial}
\newcommand{\cl}{\mathrm{Cl}}
\newcommand{\C}{\mathrm{C}}
\newcommand{\id}{\mathrm{id}}
\renewcommand{\d}{\mathrm{d}}
\newcommand{\e}{\mathrm{e}}
\newcommand{\essup}{\mathrm{ess\,sup}}
\newcommand{\op}{\mathrm{op}}
\newcommand{\tr}{\mathrm{tr}}
\newcommand{\irm}{\mathrm{i}}
\newcommand{\D}{\mathrm{D}}
\newcommand{\dv}{\mathrm{div\,}}
\newcommand{\Lap}{\mathrm{\Delta}}
\newcommand{\Hd}{\dot H}
\newcommand{\eqnb}{\begin{equation}}
\newcommand{\eqnbs}{\begin{equation*}}
\newcommand{\eqnbsa}{\begin{equation*}\begin{aligned}}
\newcommand{\eqnba}{\begin{equation}\begin{aligned}}
\newcommand{\eqnbl}[1]{\begin{equation}\label{#1}}
\newcommand{\eqnbal}[1]{\begin{equation}\label{#1}\begin{aligned}}
\newcommand{\eqnes}{\end{equation*}}
\newcommand{\eqne}{\end{equation}}
\newcommand{\eqnesa}{\end{aligned}\end{equation*}}
\newcommand{\eqnea}{\end{aligned}\end{equation}}
\newcommand{\loc}{\mathrm{loc}}
\newcommand{\re}[1]{(\ref{#1})}
\newcommand{\oneChar}{\hspace{11pt}}%insert space of width equivalent to one character 
\newcommand{\comment}[1]{}

%BB and mathcal letters
\newcommand{\RR}{\mathbb{R}}
\newcommand{\TT}{\mathbb{T}}
\newcommand{\PP}{\mathbb{P}}
\newcommand{\ZZ}{\mathbb{Z}}
\newcommand{\CC}{\mathbb{C}}
\renewcommand{\SS}{\mathbb{S}}
\newcommand{\cP}{\mathcal{P}}
\newcommand{\cA}{\mathcal{A}}
\newcommand{\cB}{\mathcal{B}}
\newcommand{\cS}{\mathcal{S}}
\newcommand{\cV}{\mathcal{V}}
\newcommand{\cI}{\mathcal{I}}
\newcommand{\NN}{\mathbb{N}}
\newcommand{\ka}{\mathfrak{a}}
\newcommand{\kA}{\mathfrak{A}}


%Theorems
\newtheorem{theorem}{Theorem}
\newtheorem{proposition}[theorem]{Proposition}%[section]
\newtheorem{definition}[theorem]{Definition}
\newtheorem{lemma}[theorem]{Lemma}%[section]
\newtheorem{corollary}[theorem]{Corollary}%[section]
\newtheorem{apxlem}{Lemma}[section]{\bf}{\it}

%\newcommand{\Method}[4]{{\tt #1}&\makecell[l{p{5cm}}]{{\bf Args: }#2\\{\bf Desc: }#3\\{\bf Return: }#4}}

%method specification as item in description environment
\newcommand{\MethodItem}[4]{\item[{\tt #1}]{\bf Args: }#2\\{\bf Desc: }#3\\{\bf Return: }#4}
%Argument description formatting to use in argument 2 of \MethodItem
\newcommand{\MethodArg}[2]{{\it #1} - #2}

%attribute specification as item in description environment
\newcommand{\AttrItem}[2]{\item[{\tt #1}] - #2}

\title{CoNxNect API v0.1}
\author{HappyGnome}
\date{March 2019}
\begin{document}
\maketitle
\tableofcontents
%######################################
\section{Introduction}
TODO
%######################################
\section{AI Player Library}
\subsection{Overview}
A player library defines an ai player class, a player factory class, and a module file which is used to load the rest. For definiteness, we will call these {\tt aiPlayer}, {\tt aiFactory} and {\tt aiModule} (i.e.\ in the file {\tt aiModule.py}), although any names may be used. In order that the library can be found by CoNxNect, {\tt aiModule.py} should be placed in {\tt <CoNxNect\_root>/AI\_Players}. Any additional files should be placed in {\tt <CoNxNect\_root>/AI\_Players/<aiModule>/}
\subsection{\tt aiModule}
An AI player module, must implement the following in global scope:
\begin{description}
\MethodItem{getFactory()}{None}{}{An instance of {\tt aiFactory}. This may be static within the module. It will only be used in the calling thread.}
\end{description}
%************************************
\subsection{{\tt aiPlayer} Reference}
%The {\tt aiPlayer} should inherit from {\tt conxnect.aiPlayer\_base}. {\tt conxnect.aiPlayer\_base} has the following 

The following methods should be implemented:
\begin{description}
%\MethodItem{\_\_init\_\_(self,spec, playerTurn)}{\MethodArg{spec}{an instance of gameSpec, defining the grid size, player count and victory conditions.}\\ }{spec and playerTurn should be passed to {\tt super.\_\_init\_\_}}{The column in which this player plays next.}
\MethodItem{makeMove(self)}{self}{The player will not be notified of its own moves, and so should update itself accordingly.}{Index of the column in which this player plays next.}
\MethodItem{notifyMove(self, playerTurn, column)}{\MethodArg{playerTurn}{the turn position of the player making the move.}\\\MethodArg{column}{the column in which the player played}}{Called each time another player makes a move.}{None}
\MethodItem{beginGame(self, playerTurn)}{\MethodArg{playerTurn}{the position of this player in turn order (1= first,\ldots)}}{Initialise resources, threads etc.\ when this is called.}{None}
\MethodItem{endGame(self)}{self}{Free resources, threads etc.\ that were created at {\tt beginGame}.}{None}
\end{description}

\subsection{{\tt aiFactory} Reference}
The following methods should be implemented:
\begin{description}
%\MethodItem{\_\_init\_\_(self,spec, playerTurn)}{\MethodArg{spec}{an instance of gameSpec, defining the grid size, player count and victory conditions.}\\ }{spec and playerTurn should be passed to {\tt super.\_\_init\_\_}}{The column in which this player plays next.}
\MethodItem{newPlayer(self, spec, allowed\_depth)}{\MethodArg{spec}{an instance of gameSpec, defining the grid size, player count and victory conditions.}\\\MethodArg{allowed\_depth}{A complexity hint. The returned player should compute at most this many levels in the decision tree.}}{Generate a new instance of {\tt aiPlayer} based on the game specification.}{The new {\tt aiPlayer} instance.}
\MethodItem{newPlayers(self, spec, allowed\_depth, n)}{\MethodArg{spec}{an instance of gameSpec}\\\MethodArg{allowed\_depth}{A complexity hint. The returned player should compute at most this many levels in the decision tree.}\\\MethodArg{n}{the number of players to generate}}{Generate {\tt n} new {\tt aiPlayer} instances, based on {\tt spec} and the current factory config.}{A list containing the new {\tt aiPlayer} instances.}

\MethodItem{loadDefaultConfig(self)}{self}{Load the default config/training data for this factory. This will be called before players are generated.}{None}
\MethodItem{updateConfig(self, player\_rank)}{\MethodArg{player\_rank}{A list of ordered pairs {\tt(player, score)}. Each {\tt player} is an instance generated by this factory, and {\tt score} is the corresponding rating (higher is better), calculated from playing several games.}}{Update the parameters of the factory when generating new players. This routine may or may not save the new config/ learning data.}{None}
\MethodItem{saveConfig(self)}{self}{Save any config/learning data, as necessary. This may be called, for example before Default config is reloaded, or the program is closed.}{None}
\end{description}
%######################################
\section{Structures}
\subsection{\tt gameSpec}
Public attributes:
\begin{description}
\AttrItem{rows}{Number of rows in the grid.}
\AttrItem{cols}{Number of columns in the grid.}
\AttrItem{playerCount}{Number of players in the game.}
\AttrItem{victoryN}{Length of streak required for victory.}
\end{description}
%**********************************
%######################################
\bibliographystyle{plain}%nat}%authordate1}
\bibliography{../Bib1}{}
\end{document}